\documentclass[12pt, letterpaper]{article}

\usepackage[utf8]{inputenc}
\usepackage[T1]{fontenc}
\usepackage[french]{babel}

\title{Travail pratique \#2 -- IFT2245}
\author{Abdelhakim Qbaich \and Rémi Langevin}
\date{\today}

\begin{document}

\maketitle

\section*{Problèmes rencontrés}
\subsection*{Mauvaise utilisation de \texttt{sizeof}}
Nous avons utilisé \texttt{sizeof} dans le code dans le but d'obtenir la taille
d'un tableau alloué dynamiquement. Or, ce n'est évidemment pas valide.
\subsection*{Mutexes}
L'oubli d'un \texttt{pthread\_mutex\_unlock} après une condition a été
particulièrement problématique. Plus d'une heure a été perdue à chasser la
source du bogue.

\section*{Surprises}
\subsection*{Simplicité de l'algorithme}
L'algorithme du banquier fut, de façon surprenante, assez simple à implémenter.
\subsection*{Attaque d'intrus}
Le serveur a été mis en ligne sur un VPS publique au port \texttt{8080}.
Certaines adresses IP russes et chinoises ont tenté l'accès.

\section*{Choix effectués}
\subsection*{Numéro du client}
Dans l'implémentation actuelle, le numéro du client ne peut être négatif
(décidé arbitrairement). Il est bien évidemment possible de choisir un numéro
tel que \textbf{322}, sans problème.
\subsection*{Allocation dynamique de la matrice}
La structure des clients étant une liste chaînée, elle permet une allocation au
fur et à mesure des clients lors de la réception d'un \texttt{INI}.
\subsection*{Écriture de fonction d'envoi et de réception}
Considérant que \texttt{send} peut par moment ne pas envoyer la totalité du
\textit{buffer}, la fonction \texttt{sendall} a été écrite, pour remédier à
cette situation.

Au niveau de la réception, une fonction \texttt{recvline} a été codée afin de
faciliter la réception des messages du serveur séparées par un caractère
\textit{new line}.
\subsection*{Aléatoire}
Plutôt que d'utiliser l'opérateur modulo dans le choix des ressources à
initialiser (\texttt{INI}) ou à demander (\texttt{REQ}), ce qui est une
mauvaise pratique, une façon plus intelligente a été employée, ce qui permet
d'obtenir des valeurs plus aléatoires.
\subsection*{Utilisation de \texttt{sprintf}, \texttt{sscanf}}
Dans le projet, nous avons privilégier l'utilisation des deux méthodes
\texttt{sprintf} et \texttt{sscanf}, pour écrire et \textit{parser} les chaînes
de caractères, respectivement. Cela a été fort utile.

\section*{Options rejetées}
\subsection*{Tableau dynamique pour l'envoi et la réception du client}
Nous avons décidé de ne pas utiliser de tableau dynamique pour les messages
d'erreurs reçus par le client, puisque nous avons écrit nos propres messages
d'erreurs et savant que la taille maximale ne peut dépasser 33 caractères, y
compris \texttt{\\0}. Pour être \textit{safe}, nous avons décidé que 64 est une
taille de tableau très raisonnable à utiliser.

Quoique relativement facile à implémenter, l'\textit{overhead} de permettre une
taille dynamique pour l'envoi de commandes du client ne valait pas la peine
d'écrire l'implémentation pour. La taille maximale a été fixée à 256 caractères.
\end{document}

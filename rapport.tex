\documentclass[11pt]{article}

\usepackage[letterpaper, margin=0.75in]{geometry}
\usepackage[utf8]{inputenc}
\usepackage[T1]{fontenc}
\usepackage[french]{babel}

\title{Travail pratique \#2 - IFT-2245}
\author{Abdelhakim Qbaich & R\'emi Langevin}
\begin{document}

\maketitle

\section*{Probl\`emes rencontr\'es}
\subsection*{}

\section*{Surprises}
\subsection*{Simplicit\'e de l'algorithme}
L'algorithme du banquier fut, de fa\c{c}on surprenante, assez simple \`a 
impl\'ementer
\section*{Choix}

\section*{Options rejet\'ees}

\subsection*{Dynamic array pour les messages d'erreurs}
Nous avons d\'ecid\'e de ne pas utilis\'e de tableau dynamique pour les messages
d'erreurs re\c{c}us par le client, car il ne devrait pas avoir de probl\'emes
pour une grandeur raisonnable.

\end{document}
